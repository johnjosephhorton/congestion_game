\documentclass[11pt]{article}

\usepackage{booktabs}
\usepackage{dcolumn} 
\usepackage{epstopdf}
\usepackage{fourier}
\usepackage{fullpage}
\usepackage{graphicx}
\usepackage{hyperref}
\usepackage{longtable} 
\usepackage{natbib}
\usepackage{rotating}
\usepackage{tabularx}
\usepackage{amsmath}
\usepackage{algorithmic} 
\usepackage{algorithm2e}

\hypersetup{
  colorlinks = TRUE,
  citecolor=blue,
  linkcolor=red,
  urlcolor=black
}

\begin{document} 

\title{How Naive Players Play Empirical Congestion Games} 

\date{\today}

\author{ John J. Horton \\ NYU Stern \footnote{ Author contact information, datasets and code are currently or will be available at \href{http://www.john-joseph-horton.com/}{http://www.john-joseph-horton.com/}. } }
\maketitle

\begin{abstract}
\noindent  I propose a simple congestion game and compare player responses to the Nash Equilibria for that game to actual choices. 
I also compare actual choices to the empirical best response, given how other players play.  \newline
\noindent JEL J01, J24, J3
\end{abstract} 

\section{Introduction}

Many phenomena can be modeled as a congestion game. 
A congestion game is one with defined players and resources and the payoff to an individual depends on how many other players also choose that resources. 
The decision to take a particular road, go to a particular park or restaurant, sending an article to a journal etc. are congestion games. 
One particularly important congestion game is the decision about which job opening to apply to. 

\section{Game Description} 
In this paper, I discuss a simple discrete congestion game: 

\begin{itemize}
\item There are $n$ players 
\item There are $I$ objects, each with a payoff of $w_i$. 
\item For each object, a winner is selected from the applicants. If an applicant wins, they are awarded $w_i$. 
\end{itemize} 

\subsection{Strategy} 
Each worker pursues a mixed strategy equilibrium that is a discrete pdf over the objects. 
For each object, their probability of application is a function of $w_i$. 
In equilibrium, the expectation over each the objects that are applied to with some positive probability must be the same. 

If there are $n$ players, each playing $p(w_i)$ for that object, the count of applications to each object is a binomial random variable, $B(p(w_i), n - 1)$.
Conditional upon applying, a worker's probability of winning is $\mathbf{E}[\frac{1}{1 + B(p(w_i), n - 1)}]$.  

\section{Experimental Design} 
I ran a hypothetical congestion game on MTurk, with $I = 5$. 

\begin{figure} 
\centering 
\caption{Game interface} 
\begin{minipage}{0.85 \linewidth}
\includegraphics[width = \linewidth]{./images/HIT_interface.png}
\end{minipage} 
\end{figure}  

\cite{smith1999wealth} had some great ideas! 


\section{Results} 

\section{Conclusion} 


\bibliographystyle{aer}
\bibliography{congestion_game.bib}

\end{document} 
